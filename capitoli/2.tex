\chapter{Livello 2 [\emph{access}]}
Una nazione civile deve offrire ai suoi cittadini \textbf{servizi di base} che permettano di vivere, muoversi, istruirsi, lavorare, socializzare, realizzarsi, ecc. Maslow ha ideato una \textbf{gerarchia di bisogni umani} in cui ogni livello descrive una lista di necessità. 

I servizi che si occupano di soddisfare questi bisogni primari sono chiari, ma per un per un cittadino \textbf{digitale} quali dovrebbero essere i servizi essenziali?
\bigbreak
Un servizio diventa pubblico quando viene legislativamente riconosciuto come tale. Ogni servizio pubblico deve essere erogato secondo questi principi:
\begin{itemize}
    \item \textbf{Doverosità}: lo Stato si fanno carico di garantire l’erogazione del servizio,
    \item \textbf{Continuità}: il servizio deve essere garantito in maniera continuativa,
    \item \textbf{Parità di trattamento}: l’accesso ai servizi deve essere  equo per tutti i cittadini,
    \item \textbf{Universalità}: i servizi devono essere garantiti senza alcun tipo di discriminazione,
    \item \textbf{Economicità}: il gestore del servizio deve conseguire un margine ragionevole di utile,
    \item \textbf{Possibilità di concessione}: lo Stato può concedere l’erogazione di alcuni servizi a enti esterni, a patto che ne garantisca l'accesso e la qualità.
\end{itemize}

\bigbreak
Lo Stato deve garantire a tutti i cittadini che i servizi, analogici o digitali che siano, debbano essere garantiti in maniera indiscriminata. 

Il fenomeno del \textbf{digital divide} misura le disuguaglianze nella disponibilità di accesso alle tecnologie digitali da parte della popolazione. In Italia, nonostante non sia uno dei paesi peggiori da questo punto di vista, questa \textbf{disuguaglianza} è presente; soprattutto se si paragonano le infrastrutture del Nord Italia con quelle del Sud Italia.

\bigbreak
Per cercare di colmare questo divario tecnologico lo Stato dovrebbe concentrare le proprie risorse in modo da avere una \textbf{copertura adeguata del cablaggio della fibra ottica}. Una buona copertura avrebbe un impatto positivo immediato: garantirebbe maggiore produttività a livello imprenditoriale e un’istruzione più efficiente.

Un altro aspetto importante potrebbe essere quello di offrire ai cittadini un \textbf{sistema di archiviazione basato su cloud} per facilitare l'archiviazione di documenti legati alla Pubblica Amministrazione.

Infine sembra ragionevole pensare che ogni cittadino debba possedere un \textbf{indirizzo di posta elettronica ufficiale} in modo da semplificare la comunicazione con la Pubblica Amministrazione.

\bigbreak
La \textbf{Net Neutrality} definisce il principio secondo cui qualunque traffico di rete dovrebbe essere trattato in maniera equa. I Provider dovrebbero garantire \textbf{trasparenza} rispetto alle policy di gestione della rete e non bloccare contenuti legali discriminando così il traffico dati.