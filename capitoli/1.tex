\chapter{Livello 1 [\emph{services}]}
La disponibilità di servizi online che rispecchiano quelli fisici, può avere un impatto significativo sui diritti di cittadinanza, soprattutto per quelli erogati dalla \textbf{Pubblica Amministrazione}.
\bigbreak
Con il termine \textbf{digitalizzazione dei servizi} si indica il processo attraverso cui un servizio, originariamente erogato in forma analogica, viene implementato e distribuito attraverso l'utilizzo di tecnologie digitali. I servizi digitali offrono vantaggi quali scalabilità e velocità, ma non offrono altrettante garanzie in termini di flessibilità. 

Per arginare questo problema è necessario non effettuare una totale migrazione in favore del digitale ma mantenere una parte del servizio in forma analogica; è fondamentale avere un meccanismo di \textbf{fallback}.
\bigbreak
L'\textbf{interoperabilità} offre innumerevoli vantaggi perché permette agli utenti  di migrare da un sistema all'altro in maniera efficace e semplice. Per rendere questa procedura realizzabile è necessario che i protocolli e il formato dei dati siano sempre gli stessi.

Se da una parte i benefici lato utente sono evidenti, dall'altra le aziende mirano ad utilizzare protocolli e formati proprietari in modo da ottenere un rapporto di dipendenza che prende il nome di \textbf{vendor lock-in}\footnote{\url{https://it.wikipedia.org/wiki/Vendor_lock-in}}. Sempre più spesso le aziende credono che la possibilità che un utente possa cambiare sistema sia un male; poiché potrebbero vedere una grossa fetta di utenza migrare verso sistemi migliori.
\bigbreak
Uno dei \textbf{vantaggi} principali dei servizi digitali è rappresentato dalla possibilità di avere un sistema estremamente scalabile. L'avvento delle \textbf{Server Farm}\footnote{\url{https://it.wikipedia.org/wiki/Server_farm}} ha permesso l'allocazione "illimitata" di risorse.
\bigbreak
Un servizio digitale deve garantire che, durante l'interazione con un utente, la raccolta e la \textbf{conservazione dei dati avvenga in maniera sicura per proteggere la privacy}. 
Garantire la privacy è un problema molto importante perché l'interazione tra utente e servizio digitale lascia \textbf{tracce importanti, indistruttibili e spesso associabili a una singola persona}\footnote{Principio di Locard \emph{digitale} della Rete: un'informazione immessa nella Rete:\begin{itemize}
    \item lascia sempre tracce,
    \item tutt'altro che esigue,
    \item indistruttibili.
\end{itemize}}.

\bigbreak
Ogni servizio digitale deve essere più inclusivo possibile; ci si deve preoccupare di \textbf{non escludere utenti con abilità fisiche limitate}. 

A tal proposito esistono alcune normative che nel corso del tempo hanno portato alle linee guida attuali per la realizzazione di siti web della Pubblica Amministrazione \textbf{accessibili}, che fino al 2014 lo erano ancora poco.
\bigbreak
I servizi sono relativistici: non possono esistere due osservatori che vedono un servizio allo stesso modo. Esistono diversi casi in cui si manifesta questa situazione di relatività:
\begin{itemize}
    \item \textbf{Price discrimination}: strategia commerciale che impone a consumatori diversi, prezzi diversi per l’acquisto dello stesso bene offerto, a seconda delle caratteristiche conosciute,
    \item \textbf{Contenuti personalizzati}: Social Media e diversi motori di ricerca mostrano contenuti personalizzati a seconda della conoscenza che hanno di un utente.
\end{itemize}