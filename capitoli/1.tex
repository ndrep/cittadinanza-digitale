\chapter{Livello 1 [\emph{services}]}
Nel livello 1 vengono presentati i \textbf{servizi digitali} e analizzate le caratteristiche principali, le problematiche e l'impatto che hanno rispetto ai diritti della cittadinanza.
Negli ultimi anni l'incremento di applicazioni e siti web in grado di sostituire o integrare i servizi offerti ai cittadini è stato notevole.
La disponibilità di servizi online che rispecchiano quelli fisici può avere un impatto significativo sui diritti di cittadinanza soprattutto per quelli erogati dalla \textbf{Pubblica Amministrazione}.
\bigbreak
Con il termine \textbf{digitalizzazione dei servizi} si indica il processo attraverso cui un servizio, originariamente erogato in forma analogica, viene implementato e distribuito attraverso l'utilizzo di tecnologie digitali. I servizi digitali offrono vantaggi quali scalabilità e velocità, ma non offrono altrettante garanzie in termini di flessibilità. 
Per arginare questo problema è necessario non effettuare una totale migrazione in favore del digitale ma mantenere una parte del servizio in forma analogica; è fondamentale avere un meccanismo di \textbf{fallback}. Con il termine \emph{fallback} si intende quel meccanismo che permette la fruizione di un servizio mediante differenti canali di accesso; un sistema di questo tipo è fondamentale perché garantisce l'accesso anche in caso di malfunzionamenti, ma soprattutto fornisce al cittadino il \textbf{diritto di scegliere} come usufruirne.

\bigbreak
L'\textbf{interoperabilità} offre innumerevoli vantaggi perché permette agli utenti  di migrare da un sistema all'altro in maniera efficace e semplice. Per rendere questa procedura realizzabile è necessario che i protocolli e il formato dei dati siano sempre gli stessi.

Se da una parte i benefici lato utente sono evidenti, dall'altra le aziende mirano ad utilizzare protocolli e formati proprietari in modo da ottenere un rapporto di dipendenza che prende il nome di \textbf{vendor lock-in}\footnote{\url{https://it.wikipedia.org/wiki/Vendor_lock-in}}. Sempre più spesso le aziende credono che la possibilità che un utente possa cambiare sistema sia un male; poiché potrebbero vedere una grossa fetta di utenza migrare verso sistemi migliori.
\bigbreak
\textbf{Scalabilità e sicurezza} sono aspetti fondamentali per un servizio digitale. Fondamentale è l'utilizzo di protocolli crittografati che garantiscono una comunicazione sicura. Un servizio digitale deve garantire che durante l'interazione con un utente la raccolta e la conservazione dei dati avvenga in maniera \textbf{sicura} per proteggere la \textbf{privacy} degli utenti. 

\bigbreak
Ogni servizio digitale deve essere più inclusivo possibile; ci si deve preoccupare di \textbf{non escludere utenti con abilità fisiche limitate}. 

A tal proposito esistono alcune normative che nel corso del tempo hanno portato alle linee guida attuali per la realizzazione di siti web della Pubblica Amministrazione \textbf{accessibili}, che fino al 2014 lo erano ancora poco.
\bigbreak
La \textbf{Relatività} e il \textbf{Principio di Locard Digitale} sono concetti fondamentali anche in questo livello perché permettono un'analisi etica dei servizi e su come questi debbano comportarsi risipetto all'interazione con gli utenti. Esistono diversi casi in cui si manifesta concretamente il concetto di relatività:
\begin{itemize}
    \item \textbf{Price discrimination}: strategia commerciale che impone ai consumatori diversi prezzi per l’acquisto dello stesso bene offerto a seconda delle caratteristiche conosciute,
    \item \textbf{Contenuti personalizzati}: Social Media e diversi motori di ricerca mostrano contenuti personalizzati a seconda della conoscenza che hanno di un utente.
\end{itemize}
Uno dei \textbf{vantaggi} principali dei servizi digitali è rappresentato dalla possibilità di avere un sistema estremamente scalabile; in tal senso l'avvento delle \textbf{Server Farm}\footnote{\url{https://it.wikipedia.org/wiki/Server_farm}} ha permesso l'allocazione "illimitata" di risorse.
Di conseguenza dato che \emph{l’interazione tra un soggetto e un servizio digitale lascia sempre tracce, tutt’altro che esigue, indistruttibili e associabili direttamente ad una persona}(principio di Locard) molte aziende conservano tali tracce in modo da utilizzarle per obiettivi e scopi non del tutto \textbf{trasparenti}.