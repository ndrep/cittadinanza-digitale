\chapter{Livello 3 [\emph{education}]}
 Lo sviluppo tecnologico ha fatto si che i dispositivi che ci circondano siano sempre più intelligenti. Questa situazione ha \textbf{migliorato} la qualità della vita, ma ha reso i dispositivi più difficili da utilizzare e complessi da capire. 

Per un \textbf{cittadino digitale} è indispensabile saper conoscere, configurare e controllare i dispositivi che lo circondano in modo da non rimanerne "sopraffatto".
Nonostante esistano molti \textbf{software open source}, che possono essere studiati a fondo mediante lettura del codice sorgente, talvolta anch'essi sono di difficile comprensione.
\bigbreak
Molte leggi vanno controcorrente rispetto al \textbf{principio di trasparenza e comprensibilità} dei prodotti; un caso particolare riguarda l'utilizzo dei \textbf{brevetti} che limitano fortemente la conoscenza e la ricerca.
Alcune proposte di legge mirano ad eliminare o indebolire l'utilizzo della crittografia mediante l'utilizzo di una chiave "\emph{passe-partout}"\footnote{\url{https://www.treccani.it/vocabolario/passe-partout/}} utilizzabile direttamente dal governo. Utilizzano la scusa della lotta al terrorismo per mascherare le vere ragioni di questa scelta. Questa soluzione presenta delle lacune evidenti:
\begin{itemize}
    \item i criminali continuerebbero a "nascondersi" continuando le attività illegali,
    \item \textbf{i cittadini perderebbero il controllo rispetto ai propri dati e il concetto di privacy subirebbe un deciso ridimensionamento}.
\end{itemize}

Di fondamentale importanza rimane il concetto che ogni cittadino dovrebbe comprendere le tecnologie in modo da non dover delegare le scelte politiche; potendo quindi partecipare attivamente nelle questioni tecnologiche.
\bigbreak
Una possibile soluzione potrebbe essere quella di sviluppare un \textbf{insegnamento orientato alla critica verso il mondo digitale}, ovvero quello che viene definito \textbf{learn to code}; poiché queste problematiche sono spesso figlie della scarsa \textbf{educazione al mondo digitale}. Talvolta causate dalla mancanza di volontà/possibilità del cittadino alla scoperta e all'assenza di strumenti tecnologici adatti.
\bigbreak
Un ruolo interessante viene ricoperto dai \textbf{movimenti che partono dal "basso"} e che prendono il nome di \textbf{grassroots}. Grassroots è un termine anglosassone che indica quei movimenti politici che nascono dall'aggregazione spontanea di gruppi di cittadini. L'idea nasce dal desiderio di organizzarsi autonomamente in modo da ottemperare a quelle mancanze istituzionali spesso causate dalla lontananza dal problema da parte dello Stato, oppure perché in disaccordo con ideologie non più condivisibili.
Il \textbf{Software Libero} fa parte di questi movimenti e denuncia la \textbf{necessita` di concedere i diritti invece che toglierli a chi usa il software}. L'ideale sarebbe quello di avere software utilizzabili per qualsiasi scopo, \textbf{che sia possibile redistribuire e di cui si possa studiare il funzionamento}.
\bigbreak
Infine, nonostante le leggi siano spesso lo strumento del "male", queste possono essere sfruttate in maniera positiva. Esistono delle iniziative, che prendono il nome di \textbf{"right to repair"}, che fanno pressione sui governi affinché questi emanino delle leggi che costringano i produttori di dispositivi elettronici a \textbf{renderli facilmente riparabili} dagli utenti in maniera autonoma.


