\chapter{Internet è ostile}
    \textbf{GNUnet} è uno stack di rete alternativo per la creazione di applicazioni distribuite sicure, decentralizzate e che preservano la privacy, che ha come obiettivo quello di sostituire il vecchio stack di protocollo \textbf{Internet} insicuro. \textbf{GNUnet} è nata come un'applicazione per la pubblicazione sicura di file ed è cresciuta fino ad includere tutti i tipi di componenti e applicazioni di protocollo di base per la creazione di un Internet GNU.\\
    I requisiti sociali attuali sono ampiamente differenti rispetto a quelli degli anni 70. \textbf{Internet} rimane una rete adatta nel campo militare, dove i sistemi di rete sono gestiti da una gerarchia di comando, mentre la situazione diventa di gran lunga meno sostenibile in un contesto civile.
    A causa delle scelte progettuali fondamentali di \textbf{Internet} il traffico in rete può essere indirizzato in maniera errata, intercettato, censurato e manipolato da router appartenenti alla rete e gestiti da \textbf{utenti malevoli}. Internet moderno si è evoluto a tal punto che, come affermato da Matthew Green, "la rete è ostile"\cite{network}. \\
    \textbf{GNUnet} rappresenta una rete profondamente diversa da Internet, che mira a rivoluzionare completamente alcuni concetti base attualmente applicati.\\
    Nei successivi capitoli verrà analizzata nel dettaglio la struttura di \textbf{GNUnet} a partire dall'analisi di strategie e soluzioni che sono state adottate con l'intento di eliminare le problematiche riscontrate nell'attuale stack di rete.