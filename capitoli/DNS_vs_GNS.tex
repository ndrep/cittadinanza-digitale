\chapter{\textit{Namespaces}}
Sebbene il sistema \textbf{Domain Name Service} (DNS), che si occupa dell'assegnamento dei nomi ai \textbf{nodi della rete}, soffra di molte problematiche riguardanti \textbf{sicurezza} e \textbf{privacy}, rappresenta la spina dorsale di tutti i servizi Internet. Il progetto \textbf{GNUnet} propone un sistema alternativo chiamato \textbf{GNU Name System} (GNS).\\
Questo sistema è stato progettato in modo da mantenere inalterata l'\textbf{esperienza utente} rispetto al \textbf{DNS}, ma sostituisce quest'ultimo con un protocollo più decentralizzato, sicuro e capace di preservare la \textbf{privacy} degli utenti\cite{GNS}. Il \textbf{GNS} permette l'associazione di un nome a un token protetto da crittografia rappresentando, per alcuni aspetti, un'alternativa ad alcune delle odierne infrastrutture a chiave pubblica, in particolare \textit{X.509} per il Web \cite{L0}. \\
Il fondamento del sistema \textbf{GNS} si basa sul fornire ad ogni utente la capacità di registrare liberamente e in modo sicuro nomi come \textbf{Top Level-Domains} e di risolvere \textit{namespaces} all'interno della \textit{TLD}.\\
Un altro aspetto centrale di questa idea è quella di fornire agli utenti la possibilità di delegare in modo sicuro il controllo su un sotto-dominio ad altri utenti. Questo meccanismo semplice ma potente è preso in prestito dal progetto di \textit{SDSI/SPKI} \cite{spki}. \textbf{GNS} non richiede né dipende da un'autorità centralizzata o affidabile, rendendo il provider di sistema indipendente. La decentralizzazione per il livello di rete si ottiene utilizzando una \textit{DHT} per consentire la distribuzione e la risoluzione delle mappature di elementi \textit{key-value}.


