\chapter{Livello 4 [\emph{transparency}]}
Il livello 4  tratta della problematica della trasparenza ed è il primo livello in cui si analizza il rapporto di comunicazione tra Istituzioni e cittadino.
Nel contesto "Cittadinanza Digitale e Tecnocivismo" si parla di \textbf{trasparenza} applicandola sia ai sistemi digitali sia ai sistemi organizzativi, che per essere trasparenti devono \textbf{rendere pubbliche} informazioni sul proprio stato interno. In particolare quando si parla di relazione tra Pubblica Amministrazione e cittadino si ha un tipo di \textbf{trasparenza "top-down"}, ovvero fatta da un ente che svolge un \textbf{servizio pubblico} verso il cittadino.
\bigbreak
La Pubblica Amministrazione rimane sempre contraria nell'essere trasparente verso i cittadini. Dare ad un cittadino la possibilità di accedere a dati oggettivi \textbf{permetterebbe a quest'ultimo di acquisire consapevolezza rispetto all'utilizzo delle risorse pubbliche e alle decisioni prese per la comunità}.

Un cittadino ha molta più \textbf{fiducia} nei confronti di un \textbf{ente trasparente}, mentre resterà restio nel credere nella bontà di quest'ultimo se non sarà trasparente rispetto al proprio operato.
\bigbreak
La struttura attraverso cui la Pubblica Amministrazione pubblica i dati in suo possesso prende il nome di \textbf{Open Data}. Purtroppo \textbf{non esiste uno standard} che uniforma le modalità attraverso cui i dati devono essere pubblicati; questo implica che la maggior parte delle volte i dati sono:
\begin{itemize}
    \item \textbf{rilasciati con un formato chiuso},
    \item \textbf{inutili},
    \item \textbf{illeggibili}.
\end{itemize}
\bigbreak
Una possibile soluzione è stata proposta da Tim Berners-Lee che si basa sull'assegnamento di un punteggio che può variare a seconda dell'\textbf{esistenza del dato, dal fatto che sia leggibile, che possieda un formato libero e che sia connesso ad altri dati con standard} \textbf{Resource Description Framework (RDF)}\footnote{\url{https://it.wikipedia.org/wiki/Resource_Description_Framework}}. 
Questo tipo di valutazione mitiga il problema da un punto di vista informatico, ma non tratta aspetti importanti come la licenza, l'aggiornamento e l'utilità.

Davies, a tal proposito, ha definito una valutazione che prende in considerazione l'utilità dei dati, il contesto, la discussione rispetto a quest'ultimi e la possibilità di integrare o modificare i dati da parte del cittadino.

\bigbreak
Queste soluzioni vanno a scontrarsi con gli\textbf{ “Webstacles”}; ovvero gli ostacoli che mettono in atto le P.A. per fare in modo che i dati non siano modificabili dall’utente finale, quindi si hanno informazioni rilasciate con formati particolari per evitare che un utente terzo possa fare qualsiasi tipo di operazione, risultando, “legalmente” trasparente, ma essendolo relativamente poco nella pratica.
Capita che i \textbf{dati vengano pubblicati in maniera frammentata}, oppure mediante l'inserimento di un \textbf{captcha} per ogni download.

Alcune soluzioni, come lo \textbf{scraping}, prevedono l'estrazione dei dati grezzi e la conseguente aggregazione ai fini di rendere le informazioni disponibili in maniera compatta e comprensibile.
Nel caso in cui i dati non siano stati pubblicati \textbf{è possibile tentarne una ricostruzione mediante la raccolta sul campo} piuttosto che attraverso l'estrazione da qualche dataset.
\bigbreak
Da un punto di vista legislativo è doveroso menzionare il \textbf{Freedom of Information Act (FOIA)} che in linea teorica darebbe il potere ai cittadini di chiedere qualsiasi informazione riguardante la Pubblica Amministrazione.