\chapter{Anonimato in rete}
All'interno di \textbf{GNUnet} il termine \textbf{EGO} viene utilizzato per indicare il fatto che all'interno di \emph{questa} rete gli utenti possono avere molteplici identità non collegabili tra loro. Il concetto di identità multiple è di fondamentale importanza perché offre ad ogni individuo la capacità di gestire la propria identità a seconda del contesto in cui si trova; per esempio potrebbe essere particolarmente utile poter gestire i propri affari con un alter ego diverso da quello con cui vengono gestite altre attività. \\
A differenza del classico instradamento basato su indirizzi \textit{IP} \textbf{GNUnet} propone una sofisticata gestione dei pacchetti \textbf{basata su chiavi} con la quale è possibile identificare all'interno della rete i vari nodi; questa soluzione ha come obiettivo quello di togliere agli \textit{ISP} la capacità di manipolare e dirottare il traffico dati. \\
La rete è decentralizzata e ogni comunicazione è criptata, questo fa si che ogni utente non possa conoscere l'indirizzo \textit{IP} del nodo con cui si sta scambiando le informazioni poiché l'interscambio viene moderato da un nodo intermedio. Il nodo che fa da tramite a sua volta può conoscere gli indirizzi \textit{IP} dei due nodi che si stanno scambiando informazioni, ma non può scoprire cosa stiano effettivamente condividendo, perché solo essi conoscono le chiavi per decriptarne il contenuto. \\
La strategia è quella di utilizzare un complesso sistema basato su \textbf{chiavi crittografiche}.
La comunicazione tra due nodi della rete avviene mediante l'ausilio di un sistema \textbf{crittografico stratificato} in cui è possibile identificare uno specifico \textit{device} all'interno della oppure la rete stessa.\\
Ipotizzando una possibile comunicazione tra due utenti, il primo per mandare un messaggio al secondo dovrà criptare quest'ultimo due volte utilizzando prima la chiave rappresentante il \textit{device} e successivamente quella rappresentante la rete a cui appartiene quest'ultimo.\\
Ogni utente all'interno della rete viene identificato mediante una chiave pubblica chiamata \textbf{EGO}. Ovviamente, come detto precedentemente, ogni utente ha la possibilità di creare più chiavi in modo da poter gestire molteplici attività tramite diversi alter ego. \\
Questo sistema risolve interamente le problematiche di sicurezza relative all'utilizzo degli attuali protocolli Internet garantendo, inoltre, una completa \textbf{neutralità} della rete; l'unico \textit{malus} è rappresentato dal fatto che l'instradamento risulterà più difficoltoso e quindi lento.


